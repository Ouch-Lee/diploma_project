% !Mode:: "TeX:UTF-8"
% !TEX program  = xelatex
\begin{中文摘要}{轮腿机器人,平衡控制,轨迹跟踪}
  本毕业设计主要设计并制作了一种二轮轮腿机器人,并建立了简化的数学模型,搭建了仿真模型,设计了控制算法并在仿真中进行测试,然后在实物原型机上进行实验。本轮腿机器人拥有6个自由度,其机械设计可靠,电气系统稳定,性能好且易于开发,机器人整体负载较大,可在后期加装传感器、机械臂、用于辅助增强稳定性的动力尾或者涵道发动机。本轮腿机器人所有电机均为准直驱电机,可以实现力控,当前采用PID或者LQR实现轨迹跟踪,后期可结合动力尾、机械臂实现全身控制算法。本轮腿控制系统采用SpeedGoat作为控制器,使用MATLAB Simulink编程,与仿真环境一致,可以实现快速的控制算法实现-仿真测试-实物测试。仿真和实物实验显示,本轮腿的机械、电气和控制系统稳定可靠,且实现了空间轨迹跟踪功能,为后续加装其他负载或实现全身力控算法打下坚实基础。 
\end{中文摘要}


\begin{英文摘要}{Wheel-legged Robot,Balance Control,Trajectory Tracking}
This Undergraduate thesis presents a prototype of a bipedal leg-wheeled robot, with corresponding simplified mathematics and simulation model. The control algorithms are designed and tested in simulation and then transferred to the real-world prototype. This prototype has 6-DOF(Degree of Freedom) with a reliable mechanical design, moreover with a easy-to-develop but powerful digital electric control system. The prototype has a considerable load performance which simplifies further development of adding sensor, powered tail and E-Jets to it in the future. All motors of the prototype are Qausi-Direct Drive(QDD) motors, which are capable of performing force control. Currently, the prototype is controlled with PID or LQR controller to perform trajectory tracking tasks. Further development of Whole-Body Control(WBC) with robotic arm or powered tail installed will be applied to this robot. This robot is controlled by a central compute unit from Speedgoat, which runs the same code of simulation in MATLAB Simulink environment. This has provided a rapid routine for realization of control algorithms from testing in simulation to testing on real-world prototype. The experiment result shows that the mechanical and electric design of the robot is stable and robust, and the robot can perform trajectory tracking tasks well. Generally, it is ready for further development of loaded with other sensors or perform WBC algorithms.
\end{英文摘要}
