% !Mode:: "TeX:UTF-8"
% !TEX program  = xelatex
% \sustechthesis\ 目前版本为 \version, \LaTeX\ 毕业论文模板项目从提出到现在已有两年了。感谢为本项目贡献代码的开发人员们:
% \begin{itemize}
%     \item 梁钰栋(南方科技大学,本科 17 级);
%     \item 张志炅(南方科技大学,本科 17 级)。
% \end{itemize}
% 以及使用本项目,并提出诸多宝贵的修改意见的使用人员们:
% \begin{itemize}
%     \item 李未晏(南方科技大学,本科 15 级);
%     \item 张尔聪(南方科技大学,本科 15 级)。
% \end{itemize}

% 此外,目前的维护者并非计算机系,可能存在对协议等的错误使用,如果你在本模板中发现任何问题,请在 GitHub 中提出 \href{https://github.com/Iydon/sustechthesis/issues}{Issues},同时也非常欢迎对代码的贡献!

衷心感谢贾振中助理教授、朱政研究副教授对本人的精心指导。老师们经常与我讨论宏观的研究方法与具体的技术细节,为课题点明了方向,他们的言传身教将使我终生受益。

在课题研究过程中,林世远师兄和朱政研究副教授对我帮助良多,本毕业设计使用的电机、电机驱动及和IMU等均是在他们的帮助下选型、调试,机械设计方面,许多都是和林世远师兄讨论确定的,IMU转EtherCAT、遥控器的PCB是在朱政研究副教授的帮助下打样的,机器人实物制作往往不是一帆风顺的,其中充满各种有挑战的工程问题,如果没有他们丰富的经验和的细心的指导,这个课题很难走到目前这一步。与我一同届的黄滨鑫同学帮助计算了轮腿杆件的一些机械参数,并辅助做了实验。

此外还要感谢实验是的师兄弟们,和支持我的朋友们,对课题的关心与建议。

作为本科生,能做出这样的毕业设计,作为本科生涯的圆满的句号,我充满喜悦与自豪,在此衷心感谢所有关心我的人。

愿大家都有光明美好的未来!
